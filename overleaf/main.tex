\documentclass[
  journal=largetwo,
  manuscript=article-type,
  year=2023,
  volume=1,
]{cup-journal}

\usepackage{amsmath}
\usepackage[nopatch]{microtype}
\usepackage{booktabs}
\usepackage{subfiles}
\usepackage{soul} % write strikethrough text


\title{Exploring the Efficacy of BERT for Improving Fake News Detection Across Different Types of Machine Learning Models}

\author{S. Socha}
\affiliation{Faculty of Economic Sciences, University of Warsaw, Warsaw, Poland}
\email[S. Socha]{s.socha2@student.uw.edu.pl}

\addbibresource{bibliography.bib}

\keywords{fake news detection, NLP, BERT} %% First letter not capped

\begin{document}

\begin{abstract}
Events in recent years such as the COVID-19 pandemic and the war in Ukraine show that disinformation has entered a new high level. The methods of disinformation and the increasingly sophisticated ways of hiding it make it necessary to have more and more powerful tools for their detection. In this work, I analyse the effectiveness of Bidirectional Encoder Representations from Transformers (BERT) for improving fake news detection across different types of machine learining algorithms. The research shows that implementing BERT should improve prediction performance. I show that... \textit{WIP (to be finished after paper is finished)...}
\end{abstract}


\section{Introduction \textit{(WIP)}}
\subfile{sections/section1_introduction}

\section{Related Work \textit{(WIP)}}
\subfile{sections/section2_relatedwork}

\section{Methodology \textit{(WIP)}}
\subfile{sections/section3_methodology}

\section{Conclusion \textit{(WIP)}}
\subfile{sections/section4_conclusion}



\printendnotes

\printbibliography

\end{document}