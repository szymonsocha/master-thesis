\documentclass[
  journal=largetwo,
  manuscript=article-type,
  year=2023,
  volume=1,
]{cup-journal}

\usepackage{amsmath}
\usepackage[nopatch]{microtype}
\usepackage{booktabs}
\usepackage{subfiles}

\title{Fake News Detection via Explainable Reinforcement Learning and BERT}

\author{S. Socha}
\affiliation{Faculty of Economic Sciences, University of Warsaw, Warsaw, Poland}
\email[S. Socha]{s.socha2@student.uw.edu.pl}

\addbibresource{bibliography.bib}

\keywords{fake news, NLP, Reinforcement Learning, BERT, XAI} %% First letter not capped

\begin{document}

\begin{abstract}
Events in recent years such as the COVID-19 pandemic and the war in Ukraine show that disinformation has entered a new high level. The methods of disinformation and the increasingly sophisticated ways of hiding it make it necessary to have more and more powerful tools for their detection. In this work, I address the topic of news classification using a novel approach that combines three state-of-the-art elements of fake news detection - Reinforcement Learning, BERT, and XAI. The research shows that both using RL and implementing BERT improve prediction performance. I show that... \textit{WIP...}
\end{abstract}


\section{Introduction \textit{(WIP)}}
\subfile{sections/section1_introduction}

\section{Related Work \textit{(WIP)}}
\subfile{sections/section2_relatedwork}

\section{Methodology \textit{(WIP)}}
\subfile{sections/section3_methodology}

\section{Conclusion \textit{(WIP)}}
\subfile{sections/section4_conclusion}



\printendnotes

\printbibliography

\end{document}