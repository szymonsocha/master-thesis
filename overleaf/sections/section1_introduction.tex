Nowadays, in the era of accelerating digitization and the transfer of media to the world of the Internet, the access to information is becoming more common and widely available. It brings numerous benefits. The information flow is unhindered and one can learn about new events almost immediately. This is due to the fact that the distribution of new information is currently free of any verification and everyone can take an active part in it. This results in many advantages. Large news agencies have lost their monopoly on sharing information. They have been superseded by social networks with numerous news profiles and independent news portals. The increase in the number of distributors makes information less prone to being biased by the pressures of lobbies. In addition, decentralized information no longer has to go through time-consuming verification processes, allowing for instant dissemination.

However, this has its downsides. Devoid of any verification of information, combined with the lack of barriers to entry, the risk of spreading false information increases. In such a situation, those who generate new information or pass it on, do not necessarily have to rely on reliable sources or have expertise in a given topic. This, coupled with the fact that any user can participate in this process of creating and transmitting new information, makes it impossible to control the quality of content by traditional methods when the volume is so high.

At the end of the road that information travels through the network is the recipient of the information. It is in their interest to verify that the information they receive is reliable and true. The recipient of information operating in an environment overloaded with information faces a very difficult task of filtering out the false ones. In addition to the desire of being well-informed, there are many other motivations for filtering out false information. An example is the COVID-19 pandemic where fake news detection was important and could have serious consequences on public health and safety. Since the very beginning of the pandemic people wanted to be well-informed to protect their health. However, along with the rapid spread of the virus, there was a sudden increase in misinformation was observed, such as belief in fake cures and conspiracy theories. This caused the problem of confusion which had a real-life consequences. The spread of fake news made it harder to control the pandemic and mitigate its impact. Thus, having a method to label misinformation could help authorities to fight fake news and improve public safety. Finding the best possible method for predicting fake news is therefore a very trending topic. Research in this area can significantly contribute to how the future of humanity will be shaped. 

In this paper, I focus on using a language model created by Google - Bidirectional Encoder Representations from Transformers (BERT) - and its use for predicting fake news. More specifically, I focus on exploring how the use of BERT flows into the performance of different types of disinformation detection models.

In the first part... \textit{WIP (to be finished later)...}