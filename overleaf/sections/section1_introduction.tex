Nowadays, in the era of accelerating digitization and the transfer of media to the world of the Internet, the access to information is becoming more common and widely available. It brings numerous benefits. The information flow is unhindered and one can learn about new events almost immediately. This is due to the fact that the distribution of new information is currently free of any verification and everyone can take an active part in it. This results in many advantages. Large news agencies have lost their monopoly on sharing information. They have been superseded by social networks with numerous news profiles and independent news portals. The increase in the number of distributors makes information less prone to being biased by the pressures of lobbies. In addition, decentralized information no longer has to go through time-consuming verification processes. This makes information spread instantly.

However, this has its downsides. Devoid of any verification of information, combined with the lack of barriers to entry, the risk of spreading false information increases. In such a situation, those who generate new information or pass it on, do not necessarily have to rely on reliable sources or have expertise in a given topic. This, coupled with the fact that any user can participate in this process of creating and transmitting new information, makes it impossible to control the quality of content by traditional methods when the volume is so high.

At the end of the road that information travels through the network is the recipient of the information. It is in his interest to verify that the information he receives is reliable and true. The recipient of information who operates in an environment overloaded with information faces a very difficult task of filtering out the false ones. In addition to the desire of being well-informed, there are many other motivations for filtering out false information. An example is the COVID-19 pandemic....
