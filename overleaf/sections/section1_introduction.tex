In the present era, marked by the rapid digitization and migration of media to the Internet, information accessibility has become increasingly pervasive, offering manifold benefits. With unobstructed information flow, individuals can stay abreast of new events in near real-time, as the distribution of information no longer entails verification hurdles and participation is open to all. Consequently, news agencies have ceded their information-sharing monopoly to social networks and independent news portals, resulting in a decentralized information landscape that is less susceptible to lobbying biases. The elimination of verification processes further enables the instant dissemination of information, providing further advantages.

The widespread availability of information in the digital age brings about certain advantages, such as the unhindered flow of information and immediate access to news. However, this phenomenon also entails some drawbacks. The lack of verification measures in conjunction with the unrestricted access to information heightens the risk of disseminating misinformation. Consequently, the sources of the new information or those sharing it need not rely on dependable sources or possess the necessary expertise on the subject matter. Furthermore, since every user is capable of contributing to the creation and transmission of new information, traditional methods of quality control become ineffectual when the volume of content is immense.

It is worth defining what false information is. In this paper, the term \textit{fake news} is defined as news that contains false or manipulative information, formulated in such a way as to resemble true information as much as possible. They can be spread through various platforms such as television, print media, online news portals, and social media platforms. Fake news usually aims to exert pressure and implant a specific view on a particular phenomenon among a certain group of people. A well-directed campaign of fake news can have a significant impact on social or political events.

% napisałbym coś w rodzaju definicji "fake news" jako świadomie nieprawdziwej informacji, tworzonej i rozpowszechnianej w celu siania zamieszania i masowego wprowadzania ludzi w bląd, wpływania na podejmowane przez nich decyzje
% Szymon: zrobione :)

At the receiving end of the information transmission network, lies the responsibility of verifying the authenticity and accuracy of the information received. The sheer volume of information available poses a challenge to the recipient, who must filter out false information to remain well-informed. Besides the desire for knowledge, there are other motivations, such as public health and safety concerns, for the detection and filtering of false information. The COVID-19 pandemic is a pertinent example where the detection of fake news was crucial, given the potential impact on public health. Misinformation, including fake cures and conspiracy theories, rapidly proliferated during the pandemic, leading to confusion and real-life consequences. The spread of such news impeded efforts to control the pandemic and its impact. Therefore, methods to identify and label misinformation can aid in the fight against fake news, thereby enhancing public safety. The search for the optimal approach for predicting fake news is a trending research area, with potential significant implications for shaping the future of humanity.

This paper centers on the examination of Bidirectional Encoder Representations from Transformers (BERT), a language model developed by Google, for the purpose of fake news prediction. The primary objective is to investigate the impact of utilizing BERT on the efficacy of various models designed for detecting disinformation.

% na pewno warto tu napisać jakie są innowacyjne elementy tego badania, jakie są cele i hipotezy badawcze (nawet jeśli potem je powtórzysz)

In the first part... \textit{WIP (to be finished later)...}