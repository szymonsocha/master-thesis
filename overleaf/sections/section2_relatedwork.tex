The detection of fake news using machine learning techniques is a widely researched problem in the literature \autocite{Pathak2020}. esearchers have proposed various approaches to address this problem, which can be categorized into three distinct types: traditional methods, deep learning methods, and extensions with language models. The primary objective of these approaches is to enhance the performance of the models to detect fake news as accurately as possible. The classical approach to detecting fake news involves analyzing text length, word frequency, and the use of n-grams \autocite{Shu2017}. Additionally, researchers have proposed using sentiment analysis to improve the performance of the model. It has been found that the sentiment of a text is associated with the type of information conveyed, indicating whether the information is misinformation or not \autocite{rubin2016fake}.

The classical approach to detecting fake news using machine learning techniques is gradually being replaced by deep learning methods. Recent research papers demonstrate that deep learning techniques outperform classical approaches in identifying fake news. For example, Wang et al. achieved superior performance by employing a model based on convolutional neural networks (CNN) as opposed to classical text classification methods \autocite{wang-2017-liar}. Similarly, Rashkin et al. reported comparable results by utilizing the Long Short-Term Memory (LSTM) model \autocite{rashkin-etal-2017-truth}. Furthermore, various machine learning algorithms were evaluated on the same dataset to compare their performance in detecting fake news. The findings indicate that deep learning approaches, such as Deep CNN, LSTM, or RNN, exhibit superior performance over traditional machine learning methods. These results are consistent across different datasets, demonstrating the overall superiority of deep learning methods in identifying fake news \autocite{IEEE2021}. 

Moreover, there is a growing body of research indicating that deep learning models can be further enhanced through the incorporation of language models \autocite{Conroy2015}. Notably, one of the most widely used language models is the Bidirectional Encoder Representations from Transformers (BERT) \autocite{Devlin2018}. BERT is a pre-trained model designed to learn the contextual relations between words in unlabeled text. There are several variants of BERT available, with RoBERTa (Robustly optimized BERT approach) being one of the most effective in terms of performance. However, RoBERTa is computationally expensive and time-consuming \autocite{Liu2019}. Another BERT variant is DistilBERT, a lighter version that requires fewer resources and is faster, but has weaker performance \autocite{Sanh2019}. This relationship between language models and deep learning also holds true for the detection of fake news. The incorporation of language models has been shown to improve the performance of existing state-of-the-art models \autocite{Khan2021, Joy2022}.

The present study centers on examining the base version of BERT as a starting point for improving the performance of deep learning models in detecting fake news. It is viewed as a reasonable compromise between RoBERTa, which exhibits superior performance but is computationally demanding, and DistilBERT, which is lighter and faster but has weaker performance.

As previously discussed, language models have been shown to be effective for text classification, including the specific case of detecting fake news. BERT, a widely used language model, has also been confirmed to be useful in this domain by previous research, as highlighted in the paper authored by \autocite{Gundapu2021}. Specifically, the study demonstrates that BERT outperforms traditional methods such as Logistic Regression, SVC, and Ridge Classifier in text classification for fake news detection.
Furthermore, a multitude of research works \autocite{Alonso-Bartolome2021, Wang2021, Singhal2019, Aljawarneh2022, Jwa2019, Yang2019} have corroborated the applicability of BERT in various approaches to detecting fake news.

According to the existing literature, it can be inferred that the incorporation of BERT leads to improved prediction results, irrespective of the model applied to it \autocite{Kaliyar2021}. Kaliyar et al. (2021) proposed a framework in which they employed BERT for word embedding, and subsequently applied diverse machine learning models such as Multinomial Naïve Bayes, Decision Tree, Random Forest, and KNN. The authors noted that the utilization of BERT led to enhanced prediction results. Alternatively, one can aim to enhance the performance of Vanilla BERT by combining it with LSTM (BERT + LSTM) \autocite{Rai2022}. 
Considering the susceptibility of neural networks to overfitting, Naïve Bayes has been observed to outperform deep learning and language models on small datasets. However, as the size of the dataset increases, deep learning models begin to surpass Naïve Bayes in fake news prediction. The utilization of language models is expected to yield improved and state-of-the-art performance \autocite{Khan2021}.

...

None of the sources in the literature review indicated that the use of BERT worsens model performance. Therefore, I form the research hypothesis that:
\vspace{0.2cm}

\noindent\textbf{H1: The use of BERT improves model performance regardless of model type}.
\vspace{0.2cm}

Apart from this, the model most often used \textit{(or perhaps recommended? to be checked)} predicting fake news together is the X model. Based on this, I form another research hypothesis that:
\vspace{0.2cm}

\noindent\textbf{H2: BERT works best with model X}
\vspace{0.2cm}

...


\noindent\rule{3cm}{0.2pt}

\textit{
Blueprint:
\begin{enumerate}
    \item \st{Fake News Detection in general}
    \item \st{Fake News Detection with BERT}
    \item \st{What models were used}
    \item \st{Are there any clues if BERT always 
        improves the performance?}
    \item With what models BERT works best?
\end{enumerate}
}

\noindent\rule{3cm}{0.2pt}